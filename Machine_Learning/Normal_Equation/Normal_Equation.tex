\documentclass[UTF8]{ctexart}

\title{掰开揉碎看Normal Equation}
\author{王凌霄}
\date{\today}

\bibliographystyle{plain}

\usepackage[greek,english]{babel}
\usepackage{amsmath}

\newcommand\Emph{\textbf}

\begin{document}

\maketitle

%前言
\paragraph{\indent Normal Equation是一种基础的最小二乘方法,本文将从线性代数的角度来分析Normal Equation(而不是从矩阵求导的角度)。}

\paragraph{\indent 很多作者在推导公式的时候有意无意的忽略了内心的思考过程,只留下漂亮的步骤。这样的文风让很多读者跟不上节奏,最后一头雾水似懂、非懂。本文将从求解方程组入手,再讲讲投影的使用,最后进入到Normal Equation的应用。我的目的是让读者对$\overrightarrow{\theta}$=$A(A^T A)^{-1}A^T$ $\overrightarrow{y}$这个公式有一个big picture,即使日后忘记了这个公式也可以快速推出。}

\section{$Ax=b$ 对付无解的问题}
\paragraph{先看一个在R2中的例子}
\begin{quote}
	例1.0 求一个常数\theta 使 \theta a = b
\end{quote}
\section{投影:完美解和最优解}
\section{Normal Equation:应用与实现}
\end{document}
